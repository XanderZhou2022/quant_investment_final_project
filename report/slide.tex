\PassOptionsToPackage{quiet}{xeCJK}
\documentclass{beamer}
\usepackage{ctex, hyperref}
\usepackage[T1]{fontenc}

% other packages
\usepackage{latexsym,amsmath,xcolor,multicol,booktabs,calligra}
\usepackage{graphicx,pstricks,listings,stackengine}
\usepackage{ShanghaiTech}

% 只在开头手动放一页目录,关闭各 section/subsection 自动目录
\AtBeginSection[]{}
\AtBeginSubsection[]{}

% defs
\def\cmd#1{\texttt{\color{red}\footnotesize $\backslash$#1}}
\def\env#1{\texttt{\color{blue}\footnotesize #1}}
\definecolor{deepblue}{rgb}{0,0,0.5}
\definecolor{deepred}{rgb}{0.6,0,0}
\definecolor{deepgreen}{rgb}{0,0.5,0}
\definecolor{halfgray}{gray}{0.55}

\lstset{
    basicstyle=\ttfamily\small,
    keywordstyle=\bfseries\color{deepblue},
    emphstyle=\ttfamily\color{deepred},    % Custom highlighting style
    stringstyle=\color{deepgreen},
    numbers=left,
    numberstyle=\small\color{halfgray},
    rulesepcolor=\color{red!20!green!20!blue!20},
    frame=shadowbox,
}
\setbeamerfont{footnote}{size=\fontsize{5pt}{3pt}}


% =========================================================
% 0. 标题与前言 (Source: ShanghaiTech Theme Template)
% =========================================================

\title{基于机器学习的动态因子轮动策略研究}
\subtitle{以上证50成分股为例:从Regime识别到LSTM动态配置}
\author{周屹玺,李红瑶,王跃众}
\institute{上海科技大学}
\date{\today}

\begin{document}

\begin{frame}
    \titlepage
\end{frame}

\begin{frame}{汇报目录}
    % 显示一级目录,隐藏下级,保持目录页简洁
    \tableofcontents[sectionstyle=show,subsectionstyle=hide]
\end{frame}

% =========================================================
% 1. 研究动机 (Source: 实际步骤.md Part 1)
% =========================================================
\section{研究动机与因子轮动的理论背景}

%------------------------------------------------
\subsection{研究动机}
%------------------------------------------------

\begin{frame}{研究动机:因子配置是否应当保持不变?}
\small
\begin{itemize}
    \item 传统多因子投资框架通常假设:
    \begin{itemize}
        \item 各类风格因子在长期内能够持续提供稳定的风险溢价;
        \item 因而在实务中多采用静态或准静态的因子权重配置方式。
    \end{itemize}
    \vspace{0.5em}
    \item 然而,近年来的理论与实证研究表明:
    \begin{itemize}
        \item 因子收益并非时间不变;
        \item 其表现呈现出明显的阶段性、周期性与结构性变化。
    \end{itemize}
\end{itemize}
\end{frame}

%------------------------------------------------

\begin{frame}{因子收益非平稳性的经济直觉}
\small
\begin{itemize}
    \item 不同因子对应不同的风险补偿机制,其有效性依赖于市场环境:
    \begin{itemize}
        \item \textbf{动量因子}依赖趋势延续,在方向性明确的市场中更为有效;
        \item \textbf{价值因子}反映均值回归,在风险溢价上升或情绪悲观阶段具备防御属性;
        \item \textbf{质量因子}强调盈利能力与稳健性,在震荡或不确定性较高的环境中表现相对占优;
        \item \textbf{低波动与规模因子}往往与风险偏好与资金结构变化密切相关。
    \end{itemize}
    \vspace{0.5em}
    \item 因此,单一或静态的因子配置方式:
    \begin{itemize}
        \item 可能在特定阶段表现良好;
        \item 却在其他阶段出现系统性失效。
    \end{itemize}
\end{itemize}
\end{frame}


\begin{frame}{因子轮动的核心思想}
\small
\begin{itemize}
    \item 因子轮动理论的基本观点:
    \begin{itemize}
        \item 因子溢价具有显著的时间变化特征;
        \item 因子表现与市场所处的宏观与风险状态密切相关。
    \end{itemize}
    \vspace{0.5em}
    \item 若能够在一定程度上刻画当前市场环境,并预测下一阶段不同因子的相对强弱:
    \begin{itemize}
        \item 则有可能通过动态调整因子权重;
        \item 在风险可控的前提下提升组合的风险调整后收益。
    \end{itemize}
\end{itemize}
\end{frame}

%------------------------------------------------

\begin{frame}{因子轮动在实践中的关键挑战}
\small
\begin{itemize}
    \item 因子轮动策略在实际应用中面临两项核心困难:
    \vspace{0.3em}
    \item \textbf{挑战一:市场状态不可直接观测}
    \begin{itemize}
        \item 市场环境的划分往往具有较强主观性;
        \item 缺乏客观、可重复的数据驱动识别方法。
    \end{itemize}
    \vspace{0.3em}
    \item \textbf{挑战二:因子收益的动态演化复杂}
    \begin{itemize}
        \item 因子收益具有显著的时序相关性与噪声特征;
        \item 简单规则或静态判断难以刻画其变化过程。
    \end{itemize}
\end{itemize}
\end{frame}


\begin{frame}{研究思路:从因子轮动到数据驱动建模}
\small
\begin{itemize}
    \item 本研究将因子轮动问题拆解为两个相互关联的子问题:
    \vspace{0.4em}
    \item \textbf{第一步:市场状态识别}
    \begin{itemize}
        \item 基于指数层面的市场特征,识别不同市场状态;
        \item 从实证角度检验因子收益的状态依赖性是否存在。
    \end{itemize}
    \vspace{0.4em}
    \item \textbf{第二步:因子收益预测与配置}
    \begin{itemize}
        \item 利用机器学习模型刻画因子收益的时间序列结构;
        \item 预测下一期各因子的相对表现,并生成动态因子权重。
    \end{itemize}
\end{itemize}
\end{frame}
%----------------------------------------------


\begin{frame}{研究对象与研究目标}
\small
\begin{itemize}
    \item 研究对象选择:
    \begin{itemize}
        \item 以上证50指数成分股作为股票池;
        \item 兼顾样本稳定性与市场代表性。
    \end{itemize}
    \vspace{0.5em}
    \item 研究目标:
    \begin{itemize}
        \item 重点关注动态因子配置在样本外阶段的表现;
        \item 检验其是否能够相较于静态多因子策略,
        \item 实现更优的风险调整后收益。
    \end{itemize}
    \vspace{0.5em}
    \item 最终目标:
    \begin{itemize}
        \item 构建一套可复现、可扩展的数据驱动因子轮动研究范式。
    \end{itemize}
\end{itemize}
\end{frame}


% =========================================================
% 2. 轮动基础验证 (Source: 实际步骤.md Part 2 & 3)
% =========================================================
\section{轮动基础:数据与市场状态实证}


% =========================
% 2.2 Data: Universe & Frequency
% =========================
\subsection{数据来源与预处理}

\begin{frame}{研究对象与数据结构:两类数据、统一月度对齐}
\begin{itemize}
    \item \textbf{股票池}:上证50指数成分股(大市值、流动性强、噪声与交易约束相对更小)。
    \item \textbf{时间频率}:月度(用于状态识别、因子收益序列建模与调仓回测)。
    \item \textbf{两类核心数据}(按月严格对齐):
    \begin{enumerate}
        \item 因子层面:\textbf{因子多空收益}时间序列(刻画因子溢价随时间变化)。
        \item 指数层面:\textbf{市场特征}(用于识别市场状态 Regime)。
    \end{enumerate}
    \item \textbf{关键一致性处理}:所有日期统一映射为对应月份的\textbf{第一天}作为时间索引,确保调仓点信息不穿越。
\end{itemize}
\end{frame}

% =========================
% 2.3 Factors: Selection
% =========================
\begin{frame}{风格因子选择:经典且可解释}

{\normalsize
\begin{itemize}
  \setlength{\itemsep}{3pt}
  \item 研究目标是验证“轮动机制”而非穷举因子,因此选择经典且经济含义清晰的风格因子:
  \begin{itemize}
    \setlength{\itemsep}{2pt}
    \item 动量 (Momentum): MOM20, MOM120, RSI\footnotemark[1]
    \item 价值 (Value): PB$^{-1}$, PE$^{-1}$, DIV\footnotemark[2]
    \item 质量 (Quality): ROE, PROFIT\_GR\footnotemark[3]
    \item 风险相关 (Risk-related): VOL, BETA\footnotemark[4]
  \end{itemize}

  \vspace{2pt}
  \item 这样做的好处:
  \begin{itemize}
    \setlength{\itemsep}{2pt}
    \item 降低高维因子带来的过拟合风险;
    \item 保留清晰的经济直觉,便于解释“为什么轮动”。
  \end{itemize}
\end{itemize}
}

% -------- Footnotes: compact & slide-friendly --------
\tiny
\setlength{\baselineskip}{0.3\baselineskip}

\footnotetext[1]{Jegadeesh, N. and Titman, S. (1993), ``Returns to Buying Winners and Selling Losers: Implications for Stock Market Efficiency,'' \emph{Journal of Finance}; Moskowitz, T.J., Ooi, Y.H. and Pedersen, L.H. (2012), ``Time Series Momentum,'' \emph{Journal of Financial Economics}; Lo, A.W., Mamaysky, H. and Wang, J. (2000), ``Foundations of Technical Analysis: Computational Algorithms, Statistical Inference, and Empirical Implementation,'' \emph{Journal of Finance}.}

\footnotetext[2]{Fama, E.F. and French, K.R. (1992), ``The Cross-Section of Expected Stock Returns,'' \emph{Journal of Finance}; Lakonishok, J., Shleifer, A. and Vishny, R.W. (1994), ``Contrarian Investment, Extrapolation, and Risk,'' \emph{Journal of Finance}.}

\footnotetext[3]{Novy-Marx, R. (2013), ``The Other Side of Value: The Gross Profitability Premium,'' \emph{Journal of Financial Economics}; Fama, E.F. and French, K.R. (2015), ``A Five-Factor Asset Pricing Model,'' \emph{Journal of Financial Economics}.}

\footnotetext[4]{Frazzini, A. and Pedersen, L.H. (2014), ``Betting Against Beta,'' \emph{Journal of Financial Economics}; Ang, A., Hodrick, R.J., Xing, Y. and Zhang, X. (2006), ``The Cross-Section of Volatility and Expected Returns,'' \emph{Journal of Finance}.}

\end{frame}



% =========================
% 2.4 Factor Return Construction
% =========================
\begin{frame}{从横截面信号到时间序列:因子多空收益(Long--Short)}
\begin{itemize}
    \item 目的:把“因子暴露”转换为可建模的\textbf{因子收益序列}。
    \item 月度调仓日:对成分股按某因子值排序,构造五分位组合(G1--G5)。
    \item 定义因子多空收益:
\end{itemize}

\vspace{0.4em}
\begin{equation*}
r^{LS}_{t,f} \;=\; r_{t,f}^{G5} \;-\; r_{t,f}^{G1}
\end{equation*}

\vspace{-0.2em}
\begin{itemize}
    \item 解释:通过多空构造在较大程度上剥离市场整体方向,\textbf{更直接度量因子溢价}的时间变化。
\end{itemize}
\end{frame}

% =========================
% 2.5 Regime features
% =========================
\subsection{市场状态(Regime)识别:从指数特征到状态标签}

\begin{frame}{市场状态信息:用指数层面特征刻画“环境”}
\begin{itemize}
    \item Regime 识别不使用个股微观信息,而使用上证50指数特征,刻画整体环境:
    \begin{itemize}
        \item 月度指数收益率(ret\_month)
        \item 20日(及扩展:60日)年化波动率(vol\_20\_annual\_month\_end 等)
        \item 月度成交量总和(vol\_month\_sum)及成交量变化率(可选扩展)
    \end{itemize}
    \item 三个维度对应直觉:
    \begin{itemize}
        \item \textbf{收益}:方向与趋势
        \item \textbf{波动}:风险强度/不确定性
        \item \textbf{成交}:交易活跃度/情绪与资金参与
    \end{itemize}
    \item 工程处理:所有特征进入模型前\textbf{标准化},避免量纲影响聚类。
\end{itemize}
\end{frame}

% =========================
% 2.6 KMeans and K selection
% =========================
\begin{frame}{Regime 划分方法:KMeans 无监督聚类 + 轮廓系数选 K}
\begin{itemize}
    \item 选择无监督聚类而非“人为划分牛熊”:
    \begin{itemize}
        \item 状态边界不预设,确保可重复与客观;
        \item 用数据自行决定哪些月份在特征空间中更相似。
    \end{itemize}
    \item 算法:KMeans 对月度特征向量聚类,输出每月状态标签 $z_t \in \{0,1,2\}$。
    \item 聚类数 $K$ 的选择:比较不同 $K$ 下的 \textbf{Silhouette Score},
    \begin{itemize}
        \item 在类内相似性与类间差异性之间权衡;
        \item 兼顾稳定性与经济解释性,最终取 $K=3$。
    \end{itemize}
\end{itemize}
\end{frame}

% =========================
% 2.7 Economic meaning of regimes
% =========================
\begin{frame}{三类市场状态的经济含义:收益--风险--成交的稳定区分}
\begin{itemize}
    \item \textbf{Regime 0:趋势上行/风险偏好较高}
    \begin{itemize}
        \item 指数月度收益均值为正;
        \item 20日年化波动率较低;
        \item 成交量温和(环境更“顺风”)。
    \end{itemize}
    \item \textbf{Regime 1:高波动下行/风险集中释放}
    \begin{itemize}
        \item 收益显著偏低(存在明显负收益月份);
        \item 波动率显著高;
        \item 成交量波动大(情绪剧烈、冲击更强)。
    \end{itemize}
    \item \textbf{Regime 2:低方向性震荡/交易活跃}
    \begin{itemize}
        \item 收益接近零或小幅波动;
        \item 波动率中等;
        \item 成交量维持高位(震荡但参与度高)。
    \end{itemize}
\end{itemize}
\end{frame}

% =========================
% 2.8 Regime-dependent factor performance
% =========================
\subsection{状态依赖性检验:不同 Regime 下因子强弱是否系统性变化}



\begin{frame}{实证发现:因子收益对 Regime 高度敏感(方向一致且可解释)}
\begin{itemize}
    \item \textbf{Regime 0(正收益、低波动)}:\textbf{动量类因子}整体更强
    \begin{itemize}
        \item MOM20 / MOM120 / RSI 的多空收益均值更高;
        \item 解释:趋势延续更显著,行为惯性与方向性更强。
    \end{itemize}
    \item \textbf{Regime 1(低收益、高波动)}:动量显著弱化,\textbf{价值与防御因子}相对占优
    \begin{itemize}
        \item 动量出现回撤或失效;
        \item PB$^{-1}$、PE$^{-1}$ 与低波动/低 Beta 更稳健;
        \item 解释:风险偏好下降,均值回归与防御暴露更有利。
    \end{itemize}
    \item \textbf{Regime 2(震荡、高成交)}:\textbf{质量类因子}更稳健
    \begin{itemize}
        \item ROE、PROFIT\_GR 风险调整后表现更友好;
        \item 解释:不确定性与震荡环境下,基本面稳健性溢价更易体现。
    \end{itemize}
\end{itemize}
\end{frame}

% =========================
% 2.9 Implication for modeling
% =========================
\begin{frame}{本节结论:为何可以进入“预测 $\rightarrow$ 动态权重”主线?}
\begin{itemize}
    \item 结论1:\textbf{市场状态可由指数特征客观识别},且三类状态在收益/风险/成交维度具有稳定区分。
    \item 结论2:\textbf{因子收益存在显著状态依赖性},不同风格因子在不同 Regime 下强弱系统性变化。
    \item 因此:
    \begin{itemize}
        \item 静态因子权重难以跨环境保持最优;
        \item 将问题转化为可学习任务:\textbf{用历史因子收益(及 Regime 辅助信息)预测下一期因子相对表现},
        \item 再据此生成可执行的\textbf{动态因子权重}。
    \end{itemize}
\end{itemize}
\end{frame}




% ======================================================
\section{因子轮动策略框架与权重生成机制}

\subsection{从因子收益预测到资产配置的整体逻辑}

\begin{frame}{核心问题:预测因子收益之后,如何形成可执行策略?}
\small
\begin{itemize}
    \item 前两部分已表明:
    \begin{itemize}
        \item 因子收益具有显著的时间变化与状态依赖性;
        \item 因子轮动在经济直觉与实证层面均具有合理性。
    \end{itemize}
    \vspace{0.5em}
    \item 进一步的问题是:
    \begin{itemize}
        \item 若能够预测下一期各因子的相对表现,
        \item 如何将预测结果转化为\textbf{连续、可控、可回测的因子配置权重}?
    \end{itemize}
    \vspace{0.5em}
    \item 本研究关注的是\textbf{因子层面的配置问题},
    \begin{itemize}
        \item 而非个股择时或择股问题。
    \end{itemize}
\end{itemize}
\end{frame}

%------------------------------------------------------

\begin{frame}{因子轮动策略的整体流程}
\small
\begin{enumerate}
    \item 基于历史数据,对下一期\textbf{各因子多空收益}进行预测;
    \item 将预测得到的因子收益向量映射为\textbf{因子配置权重};
    \item 在因子层面构建组合,并映射到股票组合进行回测。
\end{enumerate}
\vspace{0.6em}
\begin{itemize}
    \item 该流程的核心在于:
    \begin{itemize}
        \item 因子权重随时间连续变化;
        \item 不依赖人工规则或离散切换。
    \end{itemize}
\end{itemize}
\end{frame}

%------------------------------------------------------
\subsection{因子收益预测的角色定位}
%------------------------------------------------------

\begin{frame}{预测目标的定义:预测“因子相对强弱”,而非绝对收益}
\small
\begin{itemize}
    \item 本研究中,模型的预测对象为:
    \begin{itemize}
        \item 下一期十个因子的\textbf{多空收益向量}
    \end{itemize}
    \vspace{0.5em}
    \item 关注重点并非预测值本身是否精确,
    \begin{itemize}
        \item 而是因子之间的\textbf{相对排序与强弱结构}。
    \end{itemize}
    \vspace{0.5em}
    \item 因此:
    \begin{itemize}
        \item 即使存在一定预测误差,
        \item 只要相对排序具有信息含量,
        \item 就有可能改善因子配置效果。
    \end{itemize}
\end{itemize}
\end{frame}


% ======================================================
\section{模型训练数据集的构建与样本划分}
% ======================================================

%------------------------------------------------------
\subsection{最终建模数据集}
%------------------------------------------------------

\begin{frame}{最终建模数据集:统一的时间序列输入}
\small
\begin{itemize}
    \item 在完成因子多空收益与市场状态(Regime)标签合并后,
    \item 构建用于机器学习建模的\textbf{最终数据集}:
    \begin{itemize}
        \item \texttt{final\_factor\_longshort.csv}
        \item 月度频率,是后续模型训练、验证与预测的\textbf{唯一中心数据源}
    \end{itemize}
    \vspace{0.5em}
    \item 数据集中每一行对应一个月份,包含:
    \begin{itemize}
        \item 十个风格因子的多空收益(G5--G1)
        \item 当期市场状态 Regime 标签
    \end{itemize}
\end{itemize}
\end{frame}

%------------------------------------------------------

\begin{frame}{数据来源与时间范围}
\small
\begin{itemize}
    \item 最终数据集由以下两类数据按月严格对齐后构建:
    \vspace{0.4em}
    \item \textbf{因子多空收益数据}(\texttt{factor\_longshort.csv})
    \begin{itemize}
        \item 基于上证50成分股
        \item 月末横截面排序、五分位分组
        \item 计算高分组减低分组(G5$-$G1)得到因子收益
    \end{itemize}
    \vspace{0.4em}
    \item \textbf{市场状态数据}(\texttt{cluster\_info})
    \begin{itemize}
        \item 基于指数特征的 KMeans 聚类结果
        \item 为每个月分配唯一 Regime 标签
    \end{itemize}
    \vspace{0.4em}
    \item 合并后时间范围:
    \begin{itemize}
        \item \textbf{2015-01-01 至 2024-12-01}
        \item 连续、无缺失的月度观测
    \end{itemize}
\end{itemize}
\end{frame}

%------------------------------------------------------
\subsection{模型输入与输出的构造}
%------------------------------------------------------

\begin{frame}{监督学习样本的构造:滑动窗口设定}
\small
\begin{itemize}
    \item 为构造适用于时间序列模型的监督学习样本,
    \item 采用长度为 \textbf{12 个月}的滑动窗口(rolling window)。
    \vspace{0.5em}
    \item 在任意预测时点 $t$:
    \begin{itemize}
        \item \textbf{输入 $X_t$}:$t-12$ 至 $t-1$ 的历史信息
        \item \textbf{输出 $y_t$}:第 $t$ 个月的因子多空收益
    \end{itemize}
    \vspace{0.5em}
    \item 该设定确保模型在任意时点的预测,
    \begin{itemize}
        \item 仅基于当时可获得的历史数据,
        \item 符合真实投资场景下的信息约束。
    \end{itemize}
\end{itemize}
\end{frame}

%------------------------------------------------------

\begin{frame}{模型输入特征的组成}
\small
\begin{itemize}
    \item 模型输入特征由两部分在时间维度上拼接而成:
    \vspace{0.4em}
    \item \textbf{因子收益特征}
    \begin{itemize}
        \item 十个因子的历史多空收益序列
        \item 刻画因子溢价自身的时间依赖结构
    \end{itemize}
    \vspace{0.4em}
    \item \textbf{市场状态特征}
    \begin{itemize}
        \item Regime 标签经 one-hot 编码后引入
        \item 显式刻画不同市场环境下的因子表现差异
    \end{itemize}
    \vspace{0.4em}
    \item 该联合输入结构使模型能够同时利用:
    \begin{itemize}
        \item 因子内部动态信息
        \item 以及外部环境信息
    \end{itemize}
\end{itemize}
\end{frame}

%------------------------------------------------------
\subsection{样本划分与时间顺序约束}
%------------------------------------------------------

\begin{frame}{样本划分原则:严格的时间顺序切分}
\small
\begin{itemize}
    \item 金融时间序列建模中,随机划分样本会导致严重的数据泄漏风险;
    \item 本研究采用\textbf{严格按时间顺序的前推切分}。
    \vspace{0.5em}
    \item 全部监督学习样本按比例划分为:
    \begin{itemize}
        \item \textbf{训练集(70\%)}:用于模型参数学习
        \item \textbf{验证集(20\%)}:用于结构选择与训练过程监控
        \item \textbf{测试集(10\%)}:仅用于样本外预测与回测评估
    \end{itemize}
\end{itemize}
\end{frame}

%------------------------------------------------------

\begin{frame}{测试集时间区间与样本外评估}
\small
\begin{itemize}
    \item 测试集对应样本期末的时间区间:
    \begin{itemize}
        \item \textbf{2024-02-01 至 2024-12-01}
    \end{itemize}
    \vspace{0.5em}
    \item 在该区间内:
    \begin{itemize}
        \item 模型参数已完全由训练集与验证集确定;
        \item 测试集数据不参与任何形式的训练或调参;
        \item 仅用于生成因子收益预测与后续动态权重回测。
    \end{itemize}
    \item 因此,该区间构成本文结论的\textbf{核心样本外检验窗口}。
\end{itemize}
\end{frame}

%------------------------------------------------------
\subsection{数据穿越风险控制}
%------------------------------------------------------

\begin{frame}{数据穿越与前视偏差的控制说明}
\small
\begin{itemize}
    \item 本研究在数据构建与模型训练过程中不存在数据穿越或前视偏差:
    \vspace{0.4em}
    \item \textbf{特征层面}
    \begin{itemize}
        \item 因子收益与 Regime 标签均基于当期及之前信息计算;
        \item 不使用任何未来数据。
    \end{itemize}
    \vspace{0.4em}
    \item \textbf{样本划分层面}
    \begin{itemize}
        \item 训练、验证、测试集严格按时间顺序切分;
        \item 测试集不参与模型训练与参数选择。
    \end{itemize}
\end{itemize}
\end{frame}

%------------------------------------------------------

\begin{frame}{预测—配置—回测的时间一致性}
\small
\begin{itemize}
    \item 在测试集的每一个月份 $t$:
    \begin{itemize}
        \item 模型仅使用 $t-1$ 及之前 12 个月的信息进行预测;
        \item 基于预测的因子收益生成当期因子权重;
        \item 再用于 $t$ 期的策略配置与回测。
    \end{itemize}
    \vspace{0.5em}
    \item 整个流程严格遵循:
    \begin{itemize}
        \item \textbf{先预测、再配置、后实现收益}的时间顺序;
    \end{itemize}
    \item 因此,最终结果具有真实投资场景下的可解释性与可信度。
\end{itemize}
\end{frame}



% ======================================================
\section{构建因子轮动预测模型}
% ======================================================

%------------------------------------------------------
\subsection{建模目标与问题设定}
%------------------------------------------------------

\begin{frame}{建模目标:预测下一期因子相对表现}
\small
\begin{itemize}
    \item 在完成数据集构建与样本划分后,进入机器学习建模阶段。
    \vspace{0.4em}
    \item 本研究的预测目标并非:
    \begin{itemize}
        \item 个股收益,或市场方向;
    \end{itemize}
    \item 而是:
    \begin{itemize}
        \item \textbf{下一期十个风格因子的多空收益向量}。
    \end{itemize}
    \vspace{0.4em}
    \item 该设定的核心关注点在于:
    \begin{itemize}
        \item 因子之间的\textbf{相对强弱结构},
        \item 而非单一因子收益预测的绝对精度。
    \end{itemize}
\end{itemize}
\end{frame}

%------------------------------------------------------

\begin{frame}{为什么采用联合预测(Multi-output)而非独立模型?}
\small
\begin{itemize}
    \item 十个因子收益并非相互独立:
    \begin{itemize}
        \item 宏观风险偏好变化可能同时影响多类因子;
        \item 市场状态切换往往带来系统性的因子结构变化。
    \end{itemize}
    \vspace{0.5em}
    \item 若为每个因子单独训练模型:
    \begin{itemize}
        \item 忽略因子间的相关结构;
        \item 在样本量有限的月频数据下更易过拟合。
    \end{itemize}
    \vspace{0.5em}
    \item 因此,本研究采用:
    \begin{itemize}
        \item \textbf{单一模型同时预测十个因子收益}的多任务回归设定,
        \item 以提升统计效率与泛化能力。
    \end{itemize}
\end{itemize}
\end{frame}

%------------------------------------------------------
\subsection{模型结构与输入输出}
%------------------------------------------------------

\begin{frame}{模型选择:LSTM 时间序列预测框架}
\small
\begin{itemize}
    \item 因子收益序列具有以下特征:
    \begin{itemize}
        \item 显著的时间依赖性;
        \item 非平稳性与阶段性变化;
        \item 对近期历史信息更为敏感。
    \end{itemize}
    \vspace{0.4em}
    \item 因此选择 LSTM(Long Short-Term Memory)模型:
    \begin{itemize}
        \item 显式建模时间序列依赖结构;
        \item 在有限样本下相对稳健;
        \item 能够自然接收多维时序输入。
    \end{itemize}
\end{itemize}
\end{frame}

%------------------------------------------------------

\begin{frame}{模型输入输出的张量结构}
\small
\begin{itemize}
    \item 输入样本 $X_t$:
    \begin{itemize}
        \item 过去 12 个月的历史序列;
        \item 每个时间步包含:
        \begin{itemize}
            \item 10 个因子多空收益;
            \item 市场 Regime 的 one-hot 编码特征。
        \end{itemize}
    \end{itemize}
    \vspace{0.4em}
    \item 张量维度表示为:
\end{itemize}

\vspace{0.2em}
\[
X \in \mathbb{R}^{(\text{样本数},\ 12,\ \text{特征数})}
\]

\vspace{0.3em}
\begin{itemize}
    \item 输出 $y_t$:
    \begin{itemize}
        \item 第 $t$ 个月的十维因子多空收益向量;
        \item 属于典型的\textbf{多输出回归问题}。
    \end{itemize}
\end{itemize}
\end{frame}

%------------------------------------------------------
\subsection{网络结构与训练设定}
%------------------------------------------------------

\begin{frame}{LSTM 网络结构概览}
\small
\begin{itemize}
    \item 网络采用 Keras Sequential 框架构建,结构如下:
    \vspace{0.4em}
    \item \textbf{LSTM 层(64 units)}
    \begin{itemize}
        \item 输入形状:$(12,\ \text{特征数})$
        \item 将历史序列压缩为一个时序表示向量
    \end{itemize}
    \vspace{0.3em}
    \item \textbf{Dropout(0.2)}
    \begin{itemize}
        \item 用于降低过拟合风险
    \end{itemize}
    \vspace{0.3em}
    \item \textbf{Dense 层(32 units, ReLU)}
    \begin{itemize}
        \item 提升非线性表达能力
    \end{itemize}
    \vspace{0.3em}
    \item \textbf{输出层(10 units, Linear)}
    \begin{itemize}
        \item 同时输出十个因子的预测收益
    \end{itemize}
\end{itemize}
\end{frame}

%------------------------------------------------------

\begin{frame}{损失函数、优化器与训练策略}
\small
\begin{itemize}
    \item 损失函数:
    \begin{itemize}
        \item 均方误差(MSE),适用于连续型回归问题
    \end{itemize}
    \vspace{0.4em}
    \item 优化器:
    \begin{itemize}
        \item Adam(学习率 0.001),在噪声梯度环境下较为稳健
    \end{itemize}
    \vspace{0.4em}
    \item 训练策略:
    \begin{itemize}
        \item 训练轮数:80 epochs
        \item 批大小:16
        \item 同步监控验证集损失以评估泛化能力
    \end{itemize}
\end{itemize}
\end{frame}

%------------------------------------------------------
\subsection{样本外预测与结果导出}
%------------------------------------------------------

\begin{frame}{样本外预测与时间索引对齐}
\small
\begin{itemize}
    \item 模型训练完成后,在测试集上执行预测:
    \begin{itemize}
        \item 输入 $X_{\text{test}}$
        \item 输出因子收益预测矩阵 $y_{\text{pred}}$
    \end{itemize}
    \vspace{0.4em}
    \item 由于采用 12 个月滑动窗口:
    \begin{itemize}
        \item 预测结果需进行严格的时间索引对齐;
        \item 确保每一行预测对应真实的预测月份。
    \end{itemize}
    \vspace{0.4em}
    \item 最终导出:
    \begin{itemize}
        \item \texttt{factor\_prediction.csv}
        \item 结构为:month + 各因子预测列(\_pred)
    \end{itemize}
\end{itemize}
\end{frame}

%------------------------------------------------------

\begin{frame}{本节小结:模型在整体框架中的角色}
\small
\begin{itemize}
    \item 本节构建了一个:
    \begin{itemize}
        \item 基于历史因子收益与市场状态信息的
        \item 多任务时间序列预测模型。
    \end{itemize}
    \vspace{0.4em}
    \item 模型输出并不直接用于交易,
    \begin{itemize}
        \item 而是作为后续因子权重生成(softmax 映射)的输入。
    \end{itemize}
    \vspace{0.4em}
    \item 下一部分将基于预测结果:
    \begin{itemize}
        \item 构建动态因子权重,
        \item 并通过回测系统性评估因子轮动策略的有效性。
    \end{itemize}
\end{itemize}
\end{frame}





% ======================================================
% Section: Backtest Results 
% ======================================================
\section{动态因子策略回测结果}

% ------------------------------------------------------
\subsection{回测设定与对照策略}
% ------------------------------------------------------

\begin{frame}{回测设定与统一口径}
\small
\begin{itemize}
    \item 回测目标:检验“预测因子收益 $\rightarrow$ softmax 映射 $\rightarrow$ 动态因子权重”
    是否能在风险可控前提下带来实质性的收益--风险改进。
    \item 统一口径:
    \begin{itemize}
        \item 测试区间(样本外):2024-02-01 -- 2024-12-01
        \item 基准:上证50指数
        \item 调仓频率:月度
        \item 组合口径:第5组(高因子组合,多头)
    \end{itemize}
\end{itemize}
\end{frame}

\begin{frame}{对照策略定义(四策略)}
\small
\begin{itemize}
    \item 等权多因子(Equal-weight):10 个因子固定等权(各 10\%),不使用 Regime/预测。
    \item 规则配置(Rule-based):规则驱动的静态配置(不依赖机器学习预测)。
    \item 动态权重(有 Regime):使用预测 + Regime 特征,softmax 生成当期权重,月度更新。
    \item 动态权重(无 Regime):仅使用预测(不含 Regime 特征),softmax 生成当期权重,月度更新。
\end{itemize}
\end{frame}

% ------------------------------------------------------
\subsection{Executive Summary:关键指标总览}
% ------------------------------------------------------

\begin{frame}{策略核心指标总览(样本外,第5组)}
\scriptsize
\begin{table}
\centering
\resizebox{\textwidth}{!}{%
\begin{tabular}{lrrrrrrrr}
\hline
策略 & CAGR & Vol & Sharpe & Sortino & Calmar & MaxDD & DD(月) & WinRate \\
\hline
等权配置 & 21.24\% & 20.56\% & 1.0294 & 4.6733 & 5.4106 & -3.93\% & 3 & 0.50 \\
规则配置 & 23.13\% & 32.75\% & 0.7754 & 2.3725 & 1.6603 & -13.93\% & 6 & 0.50 \\
动态权重(有状态) & 36.71\% & 29.77\% & 1.1856 & 3.4556 & 3.8853 & -9.45\% & 3 & 0.60 \\
动态权重(无状态) & \textbf{40.09\%} & 26.87\% & \textbf{1.3806} & \textbf{6.3496} & \textbf{13.1028} & \textbf{-3.06\%} & \textbf{2} & \textbf{0.70} \\
\hline
\end{tabular}}
\end{table}

\vspace{0.3em}
\small
\begin{itemize}
    \item 动态权重(无状态)在 CAGR、Sharpe、Calmar 与最大回撤控制上同时占优。
\end{itemize}
\end{frame}

% ------------------------------------------------------
\subsection{累计收益与收益路径}
% ------------------------------------------------------

\begin{frame}{累计收益:期末值对比(第5组 vs 基准)}
\small
\begin{table}
\centering
\begin{tabular}{lrr}
\hline
策略 & 期末累计收益(第5组) & 基准期末累计收益 \\
\hline
等权配置 & 17.42\% & 10.03\% \\
动态权重(有状态) & 29.77\% & 10.03\% \\
动态权重(无状态) & 32.46\% & 10.03\% \\
规则配置 & 18.93\% & 10.03\% \\
\hline
\end{tabular}
\end{table}
\end{frame}

\begin{frame}{累计收益曲线(建议放图)}
\small
\begin{itemize}
    \item 建议插入:四策略(第5组)与基准的累计收益率曲线图。
\end{itemize}
\vspace{0.6em}
\centering
% TODO: Replace with your exported figure path
% \includegraphics[width=0.92\textwidth]{figs/cum_returns_compare.png}
\end{frame}

% ------------------------------------------------------
\subsection{相对基准:超额收益与信息比率}
% ------------------------------------------------------

\begin{frame}{相对基准表现(CAPM / IR)}
\scriptsize
\begin{table}
\centering
\resizebox{\textwidth}{!}{%
\begin{tabular}{lrrrrr}
\hline
策略 & 年化超额收益 & Tracking Error & IR & CAPM Alpha(年化) & Corr(基准) \\
\hline
等权配置 & 0.0909 & 0.0733 & 1.0552 & 0.0910 & \phantom{0} \\
动态权重(有状态) & 0.2455 & 0.1743 & 1.2547 & 0.2014 & \phantom{0} \\
动态权重(无状态) & \textbf{0.2794} & 0.1600 & \textbf{1.4792} & \textbf{0.2363} & \phantom{0} \\
规则配置 & 0.1098 & 0.2154 & 0.5554 & 0.0986 & \phantom{0} \\
\hline
\end{tabular}}
\end{table}

\vspace{0.2em}
\small
\begin{itemize}
    \item 动态权重(无状态)在年化超额收益与信息比率(单位跟踪误差效率)上同时领先。
\end{itemize}
\end{frame}

\begin{frame}{信息比率对比(建议放图)}
\small
\begin{itemize}
    \item 建议插入:四策略 Information Ratio 柱状图。
\end{itemize}
\vspace{0.6em}
\centering
% TODO: Replace with your exported figure path
% \includegraphics[width=0.88\textwidth]{figs/ir_bar.png}
\end{frame}

% ------------------------------------------------------
\subsection{回撤与风险控制}
% ------------------------------------------------------

\begin{frame}{回撤曲线对比(建议放图)}
\small
\begin{itemize}
    \item 建议插入:四策略(第5组)回撤曲线图,用于展示尾部风险差异与修复速度。
\end{itemize}
\vspace{0.6em}
\centering
% TODO: Replace with your exported figure path
% \includegraphics[width=0.88\textwidth]{figs/drawdown_compare.png}
\end{frame}

\begin{frame}{Sharpe 对比(建议放图)}
\small
\begin{itemize}
    \item 建议插入:四策略 Sharpe Ratio 柱状图(与上页 IR 柱状图构成“风险调整后收益 + 相对基准”双证据)。
\end{itemize}
\vspace{0.6em}
\centering
% TODO: Replace with your exported figure path
% \includegraphics[width=0.88\textwidth]{figs/sharpe_bar.png}
\end{frame}

% ------------------------------------------------------
\subsection{可实施性:换手率与单位换手收益}
% ------------------------------------------------------

\begin{frame}{换手率与单位换手收益(交易成本代理)}
\scriptsize
\begin{table}
\centering
\begin{tabular}{lrrr}
\hline
策略 & 平均换手率(\%) & 平均月度收益(\%) & 单位换手收益 \\
\hline
等权配置 & 51.1111 & 1.7640 & 0.0345 \\
动态权重(有状态) & \textbf{24.4444} & 2.9410 & \textbf{0.1203} \\
动态权重(无状态) & 42.2222 & \textbf{3.0910} & 0.0732 \\
规则配置 & 32.2222 & 2.1160 & 0.0657 \\
\hline
\end{tabular}
\end{table}

\vspace{0.2em}
\small
\begin{itemize}
    \item 含 Regime 版本在更低换手下单位效率更高;无 Regime 版本收益更高但换手相对更大。
\end{itemize}
\end{frame}

% ------------------------------------------------------
\subsection{因子层面诊断:反直觉发现(表格化呈现)}
% ------------------------------------------------------

\begin{frame}{因子层面诊断:多空收益(Top--Bottom)与 IC}
\scriptsize
\begin{table}
\centering
\resizebox{\textwidth}{!}{%
\begin{tabular}{lrrrrr}
\hline
策略 & 多空CAGR & 多空Vol & 多空Sharpe & 多空MaxDD & 多空WinRate \\
\hline
等权配置 & 0.0074 & 0.0964 & 0.1195 & -0.0445 & 0.40 \\
动态权重(有状态) & -0.2450 & 0.1678 & -1.5749 & -0.2403 & 0.30 \\
动态权重(无状态) & -0.2655 & 0.2161 & -1.3010 & -0.2506 & 0.50 \\
规则配置 & -0.0715 & 0.2393 & -0.1966 & -0.2295 & 0.40 \\
\hline
\end{tabular}}
\end{table}

\vspace{0.3em}
\begin{table}
\centering
\begin{tabular}{lrrr}
\hline
策略 & IC Mean & IC Std & IC\_IR(年化) \\
\hline
等权配置 & -0.0300 & 0.1416 & -0.7339 \\
动态权重(有状态) & -0.0878 & 0.1892 & -1.6072 \\
动态权重(无状态) & -0.0833 & 0.2575 & -1.1209 \\
规则配置 & 0.0167 & 0.2819 & 0.2048 \\
\hline
\end{tabular}
\end{table}
\end{frame}

\begin{frame}{因子区分度检验(T-Stat / P-Value)}
\scriptsize
\begin{table}
\centering
\begin{tabular}{lrr}
\hline
策略 & T\_Stat & P\_Value \\
\hline
等权配置 & \phantom{-}0.11 & 54.18 \\
动态权重(有状态) & -1.45 & \phantom{0}9.29 \\
动态权重(无状态) & -1.19 & 13.38 \\
规则配置 & -0.18 & 43.13 \\
\hline
\end{tabular}
\end{table}

\vspace{0.3em}
\small
\begin{itemize}
    \item 该组“因子层面”统计并未显示传统意义上的显著增强;
    \item 动态权重的优势更可能来自组合层面权重调整对收益路径/风险暴露的工程性改良。
\end{itemize}
\end{frame}

% ------------------------------------------------------
\subsection{结论页}
% ------------------------------------------------------

\begin{frame}{回测结论}
\small
\begin{itemize}
    \item 动态权重策略整体优于静态策略:CAGR 与 Sharpe/IR 同时提升。
    \item 最优策略为\textbf{动态权重(无状态)}:
    \begin{itemize}
        \item CAGR 40.09\%,Sharpe 1.3806,IR 1.4792,MaxDD -3.06\%。
    \end{itemize}
    \item Regime 的边际贡献在本样本外区间内\textbf{有限或不稳定}:
    \begin{itemize}
        \item 含 Regime 版本 CAGR 36.71\%,Sharpe 1.1856,IR 1.2547,且回撤更深。
    \end{itemize}
\end{itemize}
\end{frame}

% ======================================================
\section{结论与研究展望}
% ======================================================

% ------------------------------------------------------
\subsection{研究结论总结}
% ------------------------------------------------------

\begin{frame}{研究问题回顾}
\small
\begin{itemize}
    \item 本研究围绕一个核心问题展开:
    \begin{itemize}
        \item 在中国股票市场中,
        \item 因子收益具有显著时间变化特征的背景下,
        \item 是否可以通过数据驱动的方法,
        \item 构建有效的因子轮动策略?
    \end{itemize}
    \vspace{0.5em}
    \item 为回答该问题,本文构建了一套完整流程:
    \begin{itemize}
        \item 因子构建与多空收益提取;
        \item 市场状态(Regime)的无监督识别;
        \item 基于时间序列的因子收益预测;
        \item 动态因子权重映射与样本外回测评估。
    \end{itemize}
\end{itemize}
\end{frame}

% ------------------------------------------------------

\begin{frame}{核心结论一:动态因子权重在组合层面显著有效}
\small
\begin{itemize}
    \item 样本外回测结果表明:
    \begin{itemize}
        \item 基于因子收益预测并通过 softmax 映射得到的动态因子权重,
        \item 在收益水平与风险调整后表现上,
        \item 均显著优于等权配置与规则型静态策略。
    \end{itemize}
    \vspace{0.5em}
    \item 更重要的是:
    \begin{itemize}
        \item 收益提升并非以放大波动或回撤为代价;
        \item 在 Sharpe、IR、Calmar 等多维指标下均表现更优。
    \end{itemize}
\end{itemize}
\end{frame}

% ------------------------------------------------------

\begin{frame}{核心结论二:优势来自权重机制,而非因子显著性增强}
\small
\begin{itemize}
    \item 在因子层面统计检验中:
    \begin{itemize}
        \item 多空组合收益整体偏弱;
        \item IC 均值与显著性检验未显示明显改善。
    \end{itemize}
    \vspace{0.5em}
    \item 这表明:
    \begin{itemize}
        \item 动态因子策略的优势,
        \item 并非来源于单个因子预测能力的统计显著增强,
        \item 而更可能来自组合层面权重调整机制,
        \item 对收益路径与风险暴露的工程性改良。
    \end{itemize}
\end{itemize}
\end{frame}

% ------------------------------------------------------

\begin{frame}{核心结论三:Regime 方法的边际贡献具有样本依赖性}
\small
\begin{itemize}
    \item 在本研究样本区间内:
    \begin{itemize}
        \item 引入 Regime 信息并未带来稳定的边际收益提升;
        \item 含 Regime 的动态权重策略在部分指标上反而略逊。
    \end{itemize}
    \vspace{0.5em}
    \item 需要强调的是:
    \begin{itemize}
        \item 该结论并不否定 Regime 方法的理论合理性;
        \item 而是反映其在样本期较短、状态转移有限条件下,
        \item 难以稳定发挥增量信息的现实约束。
    \end{itemize}
\end{itemize}
\end{frame}

% ------------------------------------------------------
\subsection{研究局限性}
% ------------------------------------------------------

\begin{frame}{研究局限性}
\small
\begin{itemize}
    \item \textbf{样本期长度有限}
    \begin{itemize}
        \item 样本外测试区间较短;
        \item 年化指标对个别月份较为敏感。
    \end{itemize}
    \vspace{0.4em}
    \item \textbf{Regime 识别的简化假设}
    \begin{itemize}
        \item 基于有限的指数特征进行聚类;
        \item 未显式刻画宏观或政策变量。
    \end{itemize}
    \vspace{0.4em}
    \item \textbf{交易成本与市场冲击的近似处理}
    \begin{itemize}
        \item 回测主要通过换手率进行间接评估;
        \item 未显式建模冲击成本与流动性约束。
    \end{itemize}
\end{itemize}
\end{frame}

% ------------------------------------------------------
\subsection{研究展望}
% ------------------------------------------------------

\begin{frame}{未来研究方向}
\small
\begin{itemize}
    \item \textbf{样本扩展与跨市场检验}
    \begin{itemize}
        \item 延长时间跨度;
        \item 在其他指数或市场中检验稳健性。
    \end{itemize}
    \vspace{0.4em}
    \item \textbf{更精细的市场状态刻画}
    \begin{itemize}
        \item 引入宏观、流动性或情绪变量;
        \item 探索连续状态或状态转移模型。
    \end{itemize}
    \vspace{0.4em}
    \item \textbf{权重映射与风险约束的改进}
    \begin{itemize}
        \item 在 softmax 映射中引入风险或换手惩罚;
        \item 探索收益–风险联合优化的权重生成机制。
    \end{itemize}
\end{itemize}
\end{frame}

% ------------------------------------------------------

\begin{frame}{总结}
\small
\begin{itemize}
    \item 本研究实证表明:
    \begin{itemize}
        \item 即便在因子统计显著性并不突出的环境中,
        \item 数据驱动的动态权重调整,
        \item 仍可在组合层面显著改善收益–风险结构。
    \end{itemize}
    \vspace{0.5em}
    \item 该发现为:
    \begin{itemize}
        \item 因子轮动策略的工程实现,
        \item 以及机器学习在多因子投资中的应用,
        \item 提供了一种可复现、可扩展的研究范式。
    \end{itemize}
\end{itemize}
\end{frame}


\end{document}